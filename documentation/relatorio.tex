\documentclass[a4paper, 12pt]{article}
\usepackage[utf8]{inputenc}

\usepackage[normalem]{ulem}
\usepackage[lmargin=2cm, tmargin=2cm, rmargin=2cm, bmargin=2cm]{geometry}
\usepackage[onehalfspacing]{setspace}
\usepackage[T1]{fontenc}
\usepackage[brazil]{babel}
\usepackage{multicol}
\usepackage{pgf,tikz,tkz-euclide,mathrsfs}
\usepackage{bm}
\usepackage{fancyhdr}
\usepackage{graphicx}
\usepackage{lipsum}
\usepackage{multirow}
\usepackage{tikz}
\usetikzlibrary{shapes,backgrounds}
\usepackage{amsmath,amsthm,amsfonts,amssymb,dsfont,mathtools,blindtext}

\usetikzlibrary{arrows,shapes,automata,petri,positioning,calc}

\renewcommand{\contentsname}{Sumário}

\title{relatório}
\author{Bruna de Castro Pereira Valões, Rauana de Carvalho Bento}
\date{24 de Abril de 2025}

\usepackage{float}
\begin{document}
\pagenumbering{gobble}

%CAPA CAPA CAPA CAPA 

\begin{center}
   \begin{tabular}{cc}
    \multirow{3}*{\includegraphics[width=1.5cm]{UFPB_Logo.png}} & \textbf{\footnotesize UNIVERSIDADE FEDERAL DA PARAÍBA - UFPB} \\
     & \textbf{ \footnotesize CENTRO DE INFORMÁTICA - CI} \\
     & \textbf{ \footnotesize DEPARTAMENTO DE INFORMÁTICA - DI} \\
   \end{tabular} 


\vspace{5cm}
\textbf{Programação Orientada a Objetos \\
Prof Danielle Rousy \\
Projeto do jogo Ping Pong em JAVA - Relatório}




\vspace{6cm}
\textbf{Bruna de Castro Pereira Valões\\ 20230102359 \\ Rauana de Carvalho Bento\\ 20230013212}



\vspace{7cm}
\textbf{JOÃO PESSOA - PB \\
2025}

\end{center}

%CAPA CAPA CAPA CAPA 

\newpage
\pagenumbering{arabic}
\thispagestyle{empty}
\newpage

\thispagestyle{empty}
\tableofcontents
\newpage


\section{Introdução}


O projeto teve como objetivo a criação de uma versão personalizada do clássico jogo Pong, desenvolvida na linguagem Java. A ideia surgiu a partir do interesse em explorar conceitos gráficos que representassem um desafio técnico e prático ao mesmo tempo. \\
 
A escolha desse tema também se deu pelo fato de que o desenvolvimento de jogos é uma aplicação envolvente e que permite explorar diversos conceitos da programação orientada a objetos.


\section{Modelagem do Problema}

A estrutura do projeto foi pensada utilizando princípios da programação orientada a objetos, com separação clara entre responsabilidades. O projeto conta com diversas classes que se comunicam entre si para compor a lógica e a interface gráfica do jogo. \\

\begin{figure}[H]
    \centering
    \includegraphics[width=1\linewidth]{Modelagem_Classes.png}
    \caption{Diagrama de Classes do Projeto Pong em JAVA
    }
    \label{fig:enter-label}
\end{figure}

Durante o desenvolvimento do projeto, aplicamos diversos conceitos de orientação a objetos. O encapsulamento foi utilizado para proteger atributos sensíveis, com métodos get e set, como em Bola e Pontuação. \\

A herança foi usada através da classe abstrata ElementoJogo, estendida por Bola e Raquete. \\

O polimorfismo apareceu no tratamento uniforme dos elementos do jogo, utilizando a interface Movable para generalizar o comportamento de movimento. \\

A classe ElementoJogo também ilustra o uso de classes abstratas, obrigando subclasses a implementar métodos essenciais. \\
\section{Ferramentas Utilizadas} 

Para o desenvolvimento do projeto foi utilizado principalmente o editor Visual Studio Code (VSCode), por ser leve, personalizável e com bom suporte para Java. \\

O projeto foi desenvolvido utilizando a biblioteca Swing para criação da interface gráfica. A renderização dos elementos foi feita por meio da classe Graphics. Sons foram reproduzidos usando a API javax.sound.sampled. \\

A organização visual foi facilitada por JPanel, JLabel e JFrame. Além disso, foi utilizado CardLayout para alternar entre a tela inicial e o jogo. 


\section{Resultados e Considerações Finais}

O maior desafio enfrentado durante o desenvolvimento foi lidar com a interface gráfica, uma vez que era uma área pouco familiar para os integrantes do grupo. Recorremos a tutoriais online e utilizamos dimensões e elementos visuais sugeridos por esses materiais, considerando que essa parte se aproxima bastante do desenvolvimento front-end, o que ainda é novidade para muitos de nós. \\

Apesar disso, conseguimos avançar bastante na compreensão e uso dos componentes, além de utilizar ferramentas de depuração para testar o funcionamento da lógica do jogo. \\

O projeto proporcionou um grande aprendizado, tanto em termos técnicos quanto conceituais. Trabalhar com Java em um contexto visual exigiu a aplicação real dos princípios da orientação a objetos, além de reforçar a importância da organização e modularização do código.


\end{document}
